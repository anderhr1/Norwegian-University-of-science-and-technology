
\chapter{Conclusions}

We performed an extensive evaluation of the relationship between the daily IV percentage and the out-of-sample one-day-ahead forecasting performance of ARMA-GARCH. The same conclusion is derived from the evaluation of both individual stocks and portfolios created by pooling stocks according to their daily IV percentage. 

Firstly, using daily closing prices, from January 2010 to September 2017, for all the constituents of the Oslo Stock Exchange All-Share Index (OSEAX), we have showed that the one-day-ahead return of stocks and portfolios with higher daily IV percentage is more difficult to forecast in terms of magnitude and direction than stocks and portfolios with lower daily IV percentage. In other words, we found a negative relationship between daily IV percentage and forecast performance in terms of magnitude and direction. 

Secondly, we have shown that the forecasting performance of the ARMA-GARCH is good. Trading according to the forecasting model generates returns better than those obtained by following a simple buy-and-hold strategy, across all levels of daily IV percentage. 

Finally, as the  daily IV percentage of stocks and portfolios increases, the excess return generated from trading according to the forecast model compared to the buy and hold-strategy increases. The positive relationship between daily IV percentage and return generated from the forecast model, despite the negative relationship between daily IV percentage and forecast performance in terms of magnitude and direction, may be seen contradictory. However, we have shown that stocks and portfolios with higher daily IV percentage also tend to have higher economic standard deviation. As a result, it might be easier for the forecasting model to predict the bigger directional movements when economic standard deviation, and hence daily IV percentage, is greater. 