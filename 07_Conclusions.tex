
\chapter{Conclusions}
\label{Conclusions}

We have performed an extensive evaluation of the relationship between the daily IVOL percentage and the out-of-sample, one-day-ahead return forecasting performance of ARMA-GARCH. The same conclusion is derived from the evaluation of both individual stocks and equally sized, equally weighted portfolios created by pooling stocks sorted by their daily IVOL percentage. 

Firstly, using daily closing data from January 2010 to September 2017 for stocks on the Oslo Stock Exchange All-Share Index (OSEAX), we have shown that the return forecasting performance of the ARMA-GARCH is good. The performance is varying in terms of accuracy, precision and direction. However, the return forecasting model performance is strong considering the accumulated return generated by following the sign strategy, across all levels of daily IVOL percentage. That is, trading according to the prediction of the return forecasting model generated a total accumulated return of $953\%$ over the 7 years considered, which is better than the return obtained by following a simple buy-and-hold strategy.

Secondly, we have showed that the out-of-sample, one-day-ahead return of stocks and portfolios with higher daily IVOL percentage is more difficult to forecast in terms of magnitude and direction than stocks and portfolios with lower daily IVOL percentage. In other words, we found a negative relationship between daily IVOL percentage and forecast performance in terms of accuracy, precision and direction.

Finally, as the  daily IVOL percentage of stocks and portfolios increases, the excess return generated from trading according to the forecast model compared to the buy and hold-strategy increases. The positive relationship between daily IVOL percentage and return generated from the forecast model, despite the negative relationship between daily IVOL percentage and forecast performance in terms of accuracy, precision and direction, may be seen contradictory. However, we have shown that stocks and portfolios with higher daily IVOL percentage also tend to have higher economic standard deviation. As a result, one possible explanation might be that it is easier for the return forecasting model to predict the bigger, directional movements when economic standard deviation, and hence daily IVOL percentage, is greater. Other explanations might be the choice of using equally weighted portfolios, which is daily rebalanced, and daily financial data.