Before commencing any empirical work, it is essential to thoroughly review the existing literature, and the relevant articles that are found can be summarized in the literature review section. This will not only help to put the proposed research in a relevant context, but also may highlight potential problem areas, and will ensure up-to-date techniques are used and that the project is not a direct (even unintentional) copy of an already existing work. The literature review should follow the style of an extended literature reviews in a scholarly journal, and should always be CRITICAL by nature. It should comment on the relevance, value, advantages and shortcomings of the cited articles. Finally you PLACE your research into the body of literature.


\chapter{Literature Review}

\textbf{1. What is IV}
\begin{itemize}
    \item One potential explanation for the latter result is that the idiosyncratic volatility is a measure of divergence of opinion, which, as argued by Miller (1977), could lead a stock to be overvalued initially and to suffer capital losses subsequently.
\end{itemize}

