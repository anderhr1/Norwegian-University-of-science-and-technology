\chapter{Literature Review}

In this chapter we will review the literature in three sections. The two first sections, covering idiosyncratic volatility and financial forecasting models, can be perceived rather independent. The last section will try to merge the two spaces of literature and connect them to the Norwegian stock market specifically. 

\section*{Idiosyncratic volatility}

Definitions of idiosyncratic volatility vary. Malkiel and Xu (1997) defined variance of stock return minus variance of the S&P 500 index. Campbell et al. (2001) and Brandt et al. (2008) used difference of individual stock return and industry return. Recently, the residual of asset pricing model has been used as idiosyncratic volatility. Malkiel and Xu (2004) and Bali and Cakici (2008) adopted both CAPM and the Fama-French three factor model (1993). The latter model was used by Spiegel and Wang (2005), Ang et al. (2006, 2009), Fu (2009), Brockman et al. (2009), and Saryal (2009). Although these definitions differ from each other, all idiosyncratic volatility represents a part that cannot be diversified. Therefore, idiosyncratic volatility is a useful measure for the idiosyncratic risk.

using the residuals of the Fama and French Three-Factor Model.


Idiosyncratic volatility (IV) can be understood as the fluctuations in the stock price, that can be attributed to company-specific factors. These factors have little or no correlation with market risk, and can therefore be substantially mitigated from a portfolio by diversification.

To further understand the theory behind and implications of IV, we will elaborate on literature exploring the pricing of risk factors and studies done on the empirical relationship between IV and stock returns.

\subsection*{The pricing of risk factors}
The behavior of stock returns, and the factors influencing them, have been extensively covered in literature.

The Capital Asset Pricing Model (CAPM) introduced by Sharpe (1964) and Lintner (1965), is well-known for for its use to determine an asset's theoretical returns. The CAPM is a specific theory of the determinants of expected returns, and it says that all investors should hold the market portfolio to remove all idiosyncratic risk. However, according to Fu (2009), some idiosyncratic risk is priced into their portfolios, since no investor holds perfectly diversified portfolios. Therefore, the analysis of idiosyncratic risk has become an important field of study.

Several studies have tried to extend the CAPM, to price the idiosyncratic risk. Banz (1981) introduced the size effect in stocks traded on the NYSE, which was the basis for the Fama and French (1993). They developed the Three-Factor Model, taking into account market, size and growth risks. Fama and French (1993) conclude that their model explains returns better than the CAPM.

Merton (1987) introduced a model that investors require a higher expected return to compensate higher idiosyncratic volatility as investors cannot be fully diversified.

\subsection*{Relationship between idiosyncratic risk and returns}
Further, there exists several studies that concludes that idiosyncratic risk is positively correlated with stocks' future returns. One of the ideas behind this conclusion is that non-diversified investors requires a risk-premium to justify any additional risk. The foremost studies on this are Merton (1973), Levy (1978) and Xu (2002).

The relationship between IV and future returns has been extensively studied. As Fu and Schutte (2010) argues, there is divided opinion of the existence of any relationship. 

\subsubsection{Positive relationship}
For instance, Malkiel and Xu (2002) indicate the positive relationship between IV and the cross-section of expected returns. They conclude that IV is more important than firm size in explaining the cross-section of returns.

Firstly, Malkiel and Xu (1997) found that idiosyncratic volatility had a positive relationship with average annual return and a negative relationship with market capitalisation using the U.S. stocks in the S&P 500 index between 1963 and 1994.
 
Thirdly, idiosyncratic volatility had a positive relationship with expected return, and a negative relationship with liquidity (Spiegel & Wang, 2005). In their sample between 1962 and 2003, the highest idiosyncratic volatility deciles had 1.33\% more average return than the lowest deciles. In addition, the highest deciles had the smallest size and the least liquidity in their result.
 
Lastly, Malkiel and Xu (2004) found a positive relationship using CAPM and the Fama- French three factor model (1993). The idiosyncratic volatility was not only positively related to the return but also a more powerful explanatory variable than the market return, size and book-to-market ratio (Malkiel & Xu, 2004).

Kotiaho (2010) carried out a similar study and found that especially small-cap stocks have a positive relationship between their IV and expected returns.

\subsubsection{Negative relationship}

Ang et al. (2006 and 2009) found the opposite result to the previous studies that a stock with a higher idiosyncratic risk had a lower expected return using the Fama-French three factor model (1993). Firstly, they adopted the volatility index (VIX) from the Chicago Board Option Exchange as an innovation proxy to estimate price of the risk and found the aggregate market volatility was negatively priced in the stock return. The innovation estimated approximately -1% and it was statistically significant. The results were robust for their value, size, liquidity, volume, dispersion of analysts’ forecasts, and momentum effects. Following their logic, if the volatility index is priced in the asset pricing model, then idiosyncratic volatility should also be priced.

The second finding was that the idiosyncratic volatility had a negative relationship with expected return. In their cross-section test, the standard deviation of the Fama-French three factor model (1993)’s residual was defined as idiosyncratic volatility. The stocks were sorted into 5 portfolios by the previous month’s idiosyncratic risk. The highest idiosyncratic risk quintile brought in -0.02% per month and the lowest risk portfolio earned 1.06% per month more than the highest risk portfolio (Ang et al., 2006). Controlling value, size, liquidity, volume, dispersion of analysts’ forecasts and momentum effect did not affect the negative relationship (Ang et al., 2006). It was also found in outside of the U.S. in their following study (Ang et al., 2009). The authors constructed local, geographical, G7 as well as world portfolios and the finding was consistent in all the sample groups (Ang et al., 2009). However, the authors were not able to identify the reason behind their finding and called it ‘a


Contrary, many studies indicates no, or even a negative, relation at all. For instance, Ang, Hodrick, Xing, and Zhang (2009) concluded that high IV stocks experience lower future returns than its counterpart.

As we also consider the relationship between a portfolio's IV and forecasting performance, Bali and Cakici (2006) study is interesting. They indicate no relation between an equally weighted stock portfolio's returns and its IV. However, Huang, Liu, Rhee, and Zhang (2010) contest these results. Their analysis shows that, the obtained relation can be explained by short-term mean-reversion.

\section*{Return forecasting models}

Financial time series prediction deals with the task of modelling the underlying data generation process using past observations to specify a model that extrapolates the time series into the future. Due to the intrinsic difficulty, widespread applications and potential economic gains of financial forecasting, much effort was devoted in the past few decades to the model specification. In the literature, there are some classical methods that have been developed in predicting financial time series. 

\subsection*{Linear forecasting models}

In the literature, two major classes of models were studied by econometricians for the purpose of forecasting. They are the statistical time series models and structural econometric models. In conventional econometric models, the variance of the disturbance is assumed to be constant.  Among the linear time a powerful method, available in the literature for univariate time-series forecasting, known as Box and Jenkins \cite{B&J} ARMA approach on stationary time series. Despite its simplicity and versatility in modelling several types of linear relationship such as pure autoregressive (AR), pure moving average (MA) and autoregressive moving average (ARMA) series, such type of models was constrained by its linear scope. 

\subsection*{Non-linear forecasting models}

However, real-world systems are seldom linear. Many economic and financial time series such as exchange rates, stock market indices, market returns, inflation rate, etc. exhibit periods of unusual large volatility, followed by periods of relative tranquillity.
Such circumstances suggest a form of heteroskedasticity in which the variance of the disturbance depends on the size of preceding disturbance and hence the conditional variance is non-constant over the sample period. 

\subsubsection*{ARMA-GARCH}

Engle \cite{Engel} showed that it is possible to model the mean and the variance of a series simultaneously by a model named autoregressive conditional heteroskedastic (ARCH); whereas Box and Jenkins \cite{B&J} modelled only the conditional mean. Bollerslev \cite{Bollerslev} extended Engle’s \cite{Engel} original work by developing a technique that allows the conditional variance to be an ARMA process. The extended process is known as the GARCH process. Wong et al. \cite{Wetal} used the AR-GARCH in exchange rate prediction. Tang et al. \cite{Tetal} added the moving average part to the conditional mean equation and used the ARMA-GARCH model for stock price prediction. GARCH models is able to deal with common financial data time series characteristics such as thick tails, high peaks and volatility clustering, as pointed out by Mandelbrot \cite{Mantelbrot}. There are, however, some characteristics of financial time series that the GARCH model is not able to deal with. The main disadvantage of the GARCH model is that conditional variance depends on the squared value of the market shocks, which in turn means that the model is sensitive only to the absolute magnitude of the variable but not to its sign leading to a presence of a leverage effect Black \cite{Black}, which represents a negative correlation between asset returns and volatility of returns. Evidence on the forecasting ability of the GARCH model is somewhat mixed. Anderson and Bollerslev \cite{A&B} showed that the GARCH model provides good volatility forecast. Conversely, some empirical studies, Brailsford and Faff \cite{f1}, Cumby et al. \cite{f2}, Figlewski \cite{f3},  Jorion \cite{f4}\cite{f5}, McMillan et al. \cite{f6}, have showed that the GARCH model tends to give poor forecasting performances. 

\subsubsection*{Artificial Intelligence-based forecasting models}

To improve the forecasting ability over the ARMA-GARCH model, Artificial Intelligence based techniques for forecasting financial time series has been advocated. Neural network is one such technique. The aforementioned forecasting models are based on some specific assumptions, such as linearity, or on error distributions, such as normality. When using neural networks to forecast financial time series, one do not need to make specific assumption. This, and the ability to approximate any nonlinear relationship with a reasonable degree of accuracy, has made the use of neural networks a popular forecasting algorithm. However, neural network suffers from a number of weaknesses including the need for a large number of controlling parameters, difficulty in obtaining a global solution and the danger of over-fitting.

Another Artificial Intelligence-based forecasting model based on support-vector machine, was developed by Vapnik \cite{Vapnik1}\cite{Vapnik2}. Since then it has been gaining popularity due to many attractive features. While the traditional neural network implements the empirical risk minimization principle, support-vector machine implements the structural risk minimization principle which seeks to minimize the upper bound of the population risk using the concept of the Vapnik–Chervonenkis dimension, as opposed to empirical risk minimization, that minimizes the error on the in-sample estimating data. Based on the structural risk minimization principle, support-vector machine achieves a balance between the training error and generalization error, leading to better forecasting performance than traditional neural network. Selecting the best model in support-vector machine is equivalent to solving a quadratic programming, which gives support-vector machine another merit of a unique global solution. 

In this paper we will use the ARMA-GARCH to forecast the one-day-ahead return. Review of artificial Intelligence based forecasting models are included in this chapter because the concept are further discussed in Chapter \label{FutureWork}



\section*{Relationship between idiosyncratic volatility and return forecasting performance in the Norwegian stock market}

In relation to studies that focus on the Norwegian stock market


NHH master: https://brage.bibsys.no/xmlui/bitstream/handle/11250/2453295/masterthesis.PDF?sequence=1&isAllowed=y

BI master: https://brage.bibsys.no/xmlui/bitstream/handle/11250/95107/Oppgave4.pdf?sequence=1