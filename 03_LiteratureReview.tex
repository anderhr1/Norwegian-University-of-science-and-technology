\chapter{Literature Review}
\label{LR}
There has been devoted less to none attention to the relationship between idiosyncratic volatility (IVOL) and return forecasting performance in existing literature. As a result, in this chapter we will review related literature in two sections. The two sections cover IVOL and financial forecasting models. In the next chapter, we will try to connect the two literature spaces using a dynamic ARMA-GARCH return forecasting model.  

\section{Idiosyncratic Volatility}  
There exists various definitions of IVOL. Malkiel and Xu \cite{malkielxu97} defines IVOL as the variance of stock return minus the variance of the S\&P 500 index. Malkiel and Xu \cite{malkielxu04} and Bali and Cakici \cite{balicakici08} calculate the IVOL as the standard deviation of the regression residuals of the CAPM or Fama-French three factor model \cite{famafrench}. The latter model was among others also adopted by Ang et al. \cite{angetal06} \cite{angetal09} and Fu \cite{Fu}. Common for the various definitions is the description of IVOL as the fluctuations in the stock price caused by company-specific factors, with little or no correlation with predefined risk factors.

To further understand the theory behind and implications of IVOL, we will elaborate on literature exploring the pricing of risk factors and the empirical relationship between IVOL and future returns.

\subsection{The Pricing of Risk Factors}
The behaviour of stock returns, and the factors influencing them, have been extensively covered in literature. The Capital Asset Pricing Model (CAPM) presented by Sharpe \cite{sharpe} and Lintner \cite{litner}, is well-known for determining an asset's theoretical expected returns. The CAPM is a specific theory of the determinants of expected returns, and it states that investors should hold the market portfolio to remove all idiosyncratic risk. However, as Fu \cite{Fu} indicates, some idiosyncratic risk is priced since no investor holds a fully diversified portfolio. His observations, among others, has made the analysis of idiosyncratic risk an important field of study.

Several studies have tried to extend the CAPM to price the idiosyncratic risk. Banz \cite{banz} introduced the size effect in stocks traded on the NYSE, which was the basis for the Fama and French \cite{famafrench}. They developed the three-factor model, taking into account market, size and growth risks. Fama and French \cite{famafrench} conclude that their model explains returns better than the CAPM. However, Merton \cite{merton87} presented a theory that investors demand a higher expected return in compensation of higher idiosyncratic risk, as investors are not totally diversified. This layed the ground for the discovering the relationship between IVOL and returns.

\subsection{Relationship Between Idiosyncratic Risk and Returns}
The relationship between IVOL and future returns has been extensively studied. Literature shows that there is divided opinion of the existence of any relationship. There exists studies indicating a positive relationships, while others advocate for a negative one. We will now highlight results from both sides.

\subsubsection{Advocates of a Positive Relationship}
There exist several studies concluding that IVOL is positively correlated with stocks' future returns. This conclusion comes from the idea of Merton \cite{merton87}, that non-diversified investors requires higher expected returns to justify increased risk. The foremost studies on this are Merton \cite{merton73} and Levy \cite{levy}.

Malkiel and Xu \cite{malkielxu02} indicate the positive relationship between IVOL and the cross-section of expected returns. They argue that IVOL is more important than firm size in explaining the cross-section of returns. Malkiel and Xu \cite{malkielxu04} found a positive relationship using CAPM and the Fama French three factor model \cite{famafrench}. The IVOL was not only positively related to the return but also a more powerful explanatory variable than the market return, size and value factors.
 
Additionally, Spiegel and Wang \cite{spiegelwang} found that IVOL had a positive relationship with expected returns, and a negative relationship with liquidity. In their sample the highest IVOL deciles had 1.33\% more average return than the lowest deciles. In addition, the highest deciles had the smallest size and the least liquidity in their result. 

\subsubsection{Advocates of a Negative Relationship}
A positive relationship between risk and return is widely accepted in financial theory. Hence, if IVOL is interpreted as firm-specific risk for an investor, it commands a premium. However, in contrast to previous studies, in recent studies this interpretation has taken a new direction.

Ang et al. \cite{angetal09} presented results concluding that stocks with a higher IVOL earned lower expected returns, using the Fama-French three factor model \cite{famafrench}. Using data from 23 countries (including Norway), they examined the cross-sectional relationship between IVOL and expected returns. In a previous article, Ang et al. \cite{angetal06} indicated a negative relationship between a stock's monthly returns and its 1-month lagged idiosyncratic risk. 

In our study we consider the relationship between a portfolio's daily IVOL percentage and forecasting performance. Thus, Bali and Cakici \cite{balicakici06} study is interesting. They indicate no relation between value-weighted portfolio returns and the value-weighted average IVOL. However, Huang et al. \cite{huang} contest the results of both Bali and Cakici \cite{balicakici06} and Ang et al. \cite{angetal09}, by showing that there is a positive relation, and it can be explained by short-term mean-reversion.

Jiang et al. \cite{jiangetal} identifies some reasons for this unexpected reality, commonly defined as the idiosyncratic volatility anomaly. They argue that IVOL is a proxy of future earnings shocks and that the anomaly to a large extent is caused by corporate selective disclosure. 

In summary, there exists divided opinion on the relationship between idiosyncratic risk and stock returns. On the positive side, literature explains the positive relationship by the fact that investors cannot fully diversify their portfolio due to the information and transaction costs. Contrary, Ang et al. \cite{angetal06} among others, found a negative relationship between IVOL and expected return. Jiang et al. \cite{jiangetal} explains the anomaly by corporate selective disclosure.

\subsubsection{Studies on the Relationship Between Idiosyncratic Volatility and Returns in the Norwegian Stock Market}

In the case of the Norwegian stock market, some studies have already analyzed the relationship between idiosyncratic risk and future returns. They focus to a large extent on the low idiosyncratic volatility anomaly introduced by Ang et al. \cite{angetal06}, which states that stocks with low IVOL tend to outperform stocks with high IVOL. 

Arnesen and Borge \cite{arnborge} examined the relation between IVOL and stock returns at the Oslo Stock Exchange. They find that stocks with low IVOL significantly outperform stocks with high IVOL in terms of Fama and French \cite{famafrench} alphas. They argue that firm size, skewness and illiquidity effects can explain the low returns of stocks with high IVOL.

Furthermore, Tjaum and Wiedswang \cite{thaumwiedswang} contribute with a measure of arbitrage activity for the idiosyncratic volatility strategy, which goes long stocks with low IVOL and short stocks with high IVOL. They conclude that the the low idiosyncratic volatility anomaly exist, but is decreasing with time and arbitrage activity. 

Contrary to the contributions above, Østnes and Hafskjær \cite{ostnes} document that the strong performance of low volatility stocks relative to high volatility stocks is not present in Norway. Their findings are robust for exposure to size, liquidity, momentum and book-to-market effects. They conclude that there is no low idiosyncratic volatility anomaly in Norway.


\section{Return Forecasting Models}

Financial time series prediction deals with the task of modelling the underlying data generation process using past observations to specify a model that extrapolates the time series into the future. Due to the difficulty, widespread applications and potential economic gains of financial forecasting, much effort has been devoted in the past few decades to financial forecasting model specification. In the literature, there are some classical methods that have been developed in predicting financial time series. 

\subsection{Linear Forecasting Models}

In conventional econometric models, the variance of the disturbance is assumed to be constant. This leads to the applications of linear forecasting models. Among the linear forecasting models, we find the popular Box and Jenkins ARMA approach. This is used for univariate, stationary time series \cite{B&J}. Despite linear forecasting model's simplicity and versatility in modelling several types of linear relationships, such as pure autoregressive (AR), pure moving average (MA) and autoregressive moving average (ARMA) series, these models are constrained by their linear scope.

\subsection{Non-linear Forecasting Models}
Real-world systems are seldom linear. Many economic and financial time series such as exchange rates, stock market indices, market returns, inflation rate, etc. exhibit periods of unusual large volatility, followed by periods of relative tranquillity. Such circumstances suggest a form of heteroskedasticity in which the variance of the disturbance depends on the size of preceding disturbance and hence the conditional variance is non-constant over the sample period. Matias and Reboredo \cite{M&R} finds that non-linear return forecasting models outperforms the linear forecasting models.

\subsubsection{ARMA-GARCH}
Engle \cite{Engel} showed that it is possible to model the mean and the variance of a series simultaneously by a model named autoregressive conditional heteroskedastic (ARCH), whereas Box and Jenkins \cite{B&J} modelled only the conditional mean. Bollerslev \cite{Bollerslev} extended Engle’s \cite{Engel} original work by developing a technique that allows the conditional variance to be an ARMA process. The extended process is known as the GARCH process. 

Following this work, Wong et al. \cite{Wetal} used the AR-GARCH in exchange rate prediction. Tang et al. \cite{Tetal} added the moving average part to the conditional mean equation and used the ARMA-GARCH model for stock price prediction. GARCH models are able to deal with common financial data time series characteristics such as thick tails, high peaks and volatility clustering, as pointed out by Mandelbrot \cite{Mandelbrot}. However, there are some characteristics of financial time series that the GARCH model is not able to deal with. The main disadvantage of the GARCH model is that conditional variance depends on the squared value of the market shocks, which in turn means that the model is sensitive only to the absolute magnitude of the variable but not to its sign leading to a presence of a leverage effect, identified by Black \cite{Black}. This represents a negative correlation between asset returns and volatility of returns. Evidence on the forecasting ability of the ARMA-GARCH model is somewhat mixed. Anderson and Bollerslev \cite{A&B} showed that the ARMA-GARCH model provides good return and volatility forecasts. Conversely, studies from Brailsford and Faff \cite{f1}, Cumby et al. \cite{f2}, Figlewski \cite{f3}, Jorion \cite{f4}\cite{f5} and McMillan et al. \cite{f6} show that the ARMA-GARCH model tends to exhibit a poor forecasting performance. 

\subsubsection{Artificial Intelligence-Based Forecasting Models}

To improve the forecasting ability of the ARMA-GARCH model, artificial intelligence-based techniques for forecasting financial time series has been introduced. Neural network is one such technique. The aforementioned forecasting models are based on some specific assumptions such as linearity or a normally distributed error process. When using neural networks to forecast financial time series, one do not need to make specific assumption. This, and the ability to approximate any nonlinear relationship with a reasonable degree of accuracy, has made the use of neural networks a popular forecasting algorithm. However, neural networks suffers from a number of weaknesses including the need for a large number of controlling parameters, difficulty in obtaining a global solution and the danger of over-fitting.

Another artificial intelligence-based forecasting model based on support-vector machine, was developed by Vapnik \cite{Vapnik1}\cite{Vapnik2}. Since then, it has been gaining popularity due to many attractive features. While the traditional neural network implements the empirical risk minimization principle, support-vector machine implements the structural risk minimization principle which seeks to minimize the upper bound of the population risk using the concept of the Vapnik–Chervonenkis dimension, as opposed to empirical risk minimization, that minimizes the error on the in-sample estimation data. Based on the structural risk minimization principle, support-vector machine achieves a balance between the training error and generalization error, leading to better forecasting performance than traditional neural networks. 

In this paper we will use the ARMA-GARCH to forecast the out-of-sample, one-day-ahead return. Review of artificial intelligence-based forecasting models is included in this chapter because the concept is relevant for future works, discussed in Chapter \ref{FutureWork}.

In contrast to existing literature on IVOL and forecasting models, we will seek to contribute to existing literature by examining the relationship between the IVOL percentage and return forecasting performance. 
