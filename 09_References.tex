\addcontentsline{toc}{chapter}{Bibliography}
\begin{thebibliography}{9}

\bibitem{famafrench} 
Eugene Fama and Kenneth French, \textit{Common risk factors in the returns on stocks and bonds}, Chicago, 1992.

\bibitem{malkielxu97} 
B.G. Malkiel and Y. Xu, \textit{Idiosyncratic Risk and Security Returns}, Princeton, 1997.

\bibitem{malkielxu04} 
B.G. Malkiel and Y. Xu, \textit{Idiosyncratic Risk and Security Returns}, Princeton, 2004.

\bibitem{malkielxu02} 
B.G. Malkiel and Y. Xu, \textit{Idiosyncratic Risk and Security Returns}, Princeton, 2002.

\bibitem{balicakici08} 
T. G. Bali and N. Cakici, \textit{Idiosyncratic Volatility and the Cross Section of
Expected Returns}, J. Financial and Quantitative analysis. 50, pp. 29-58, 2008.

\bibitem{angetal06} 
A. Ang, R. Hodrick, Y. Xing, X. Zhang, \textit{The Cross-Section of Volatility and Expected Returns}, J. of Finance, 2006.

\bibitem{angetal09} 
A. Ang, R. Hodrick, Y. Xing, X. Zhang, \textit{High idiosyncratic volatility and low returns: International and further U.S. evidence}, J. of Financial Economics, pp. 1-23, 2009.

\bibitem{Fu}
F. Fu, \textit{Idiosyncratic risk and the cross-section of expected stock returns}, Journal of Financial Economics, Volume 91, Issue 1, 2009, pp. 24-37.

\bibitem{sharpe} 
W. F. Sharpe, \textit{CAPITAL ASSET PRICES: A THEORY OF MARKET
EQUILIBRIUM UNDER CONDITIONS OF RISK}, J. of Finance, 1964.

\bibitem{litner} 
J. Lintner, \textit{The valuation of risk assets and the selection of risky investments in stock portfolios and capital budgets}, The Review of Economics and Statistics (Feb., 1965), pp. 13-37.

\bibitem{banz} 
R. W. Banz, \textit{The relationship between return and market value of common stocks}, Journal of Financial Economics (Mar., 1981), pp. 3-18.

\bibitem{merton87} 
R. C. Merton, \textit{A Simple Model of Capital Market Equilibrium with Incomplete Information}, The Journal of Finance (Jul., 1987), pp. 483-510.

\bibitem{fuschutte} 
B.G. Malkiel and Y. Xu, \textit{Idiosyncratic Risk and Security Returns}, Princeton, 1997.

\bibitem{merton73} 
R. C. Merton, \textit{An Intertemporal Capital Asset Pricing Model}, Econometrica (Sep., 1973).

\bibitem{levy} 
H. Levy, \textit{Equilibrium in an Imperfect Market: A Constraint on the Number of Securities in the Portfolio}, American Economic Review (vol. 68, 1978), pp. 643-58

\bibitem{spiegelwang} 
M. Spiegel and X. Wang, \textit{Cross-sectional Variation in Stock Returns:
Liquidity and Idiosyncratic Risk }, 2005.

\bibitem{kotiaho} 
H. Kotiaho, \textit{Idiosyncratic Risk, Financial Distress
and the Cross Section of Stock Returns}, 2010.


\bibitem{balicakici06} 
T. G. Bali and N. Cakici, \textit{Aggregate Idiosyncratic Risk and Market Returns}, Journal of Investment Management (Vol. 4., 2006).

\bibitem{huang} 
W. Huang, Q. Liu, S. G. Rhee, L. Zhang, \textit{Article Navigation
Return Reversals, Idiosyncratic Risk, and Expected Returns}, The Review of Financial Studies (Vol. 23., Jan., 2006).

\bibitem{arnborge} 
M. N. Arnesen and Ø. K. Borge, \textit{Explanations for the Low Volatility Anomaly: An Empirical Analysis of the Norwegian Stock Market}, 2017.

\bibitem{thaumwiedswang} 
C. A. Tjaum and S. Wiedswang, \textit{The Effect of Arbitrage Activity in
Low Volatility Strategies}, 2017.

\bibitem{ostnes} 
K. Østnes and H. Hafskjær, \textit{The Low Volatility Puzzle:
Norwegian Evidence}, 2013.


\bibitem{B&J} 
G.E.P. Box and G.M. Jenkins, \textit{Time Series Analysis Forecasting and Control}, Holden-Day, San Francisco, 1976.

\bibitem{Engel}
R.F. Engle, \textit{Autoregressive conditional heteroscedasticity with estimates of the variance of United Kingdom inflation}, Econometrica 50 (1982), pp. 987–1007.

\bibitem{Bollerslev}
T. Bollerslev, \textit{Generalized autoregressive conditional heteroscedasticity}, J. Econ. 31 (1986), pp. 307–327.

\bibitem{Wetal}
W.C. Wong, F. Yip, and L. Xu, \textit{Financial Prediction by Finite Mixture GARCH Model}, Proceeding of Fifth International Conference on Neural Information Processing, 1998, pp. 1351–1354.

\bibitem{Tetal}
H. Tang, K.C. Chun, and L. Xu, \textit{Finite Mixture of ARMA-GARCH Model for Stock Price Prediction}, Proceedings of the 3rd International Workshop on Computational Intelligence in Economics and Finance (CIEF2003), North
Carolina, USA, 2003, pp. 1112–1119.

\bibitem{Mandelbrot}
Mandelbrot, B., \textit{The variation of certain speculative prices}, Journal of Business 36, 1963, pp. 394–419.

\bibitem{Black}
Black, F. and Litterman, R.,\textit{Global portfolio optimization}, Financial Analysts Journal, 48(5),1992, pp. 28–43.

\bibitem{A&B}
T.G.Andersen and T. Bollerslev,\textit{Answering the skeptics: Yes, standard volatility models do provide accurate forecasts}, Int. Econ. Rev. 39 (1998), pp. 885–905.

\bibitem{f1}
T.J. Brailsford and R.W. Faff, \textit{An evaluation of volatility forecasting techniques}, J. Bank. Financ. 20 (1996), pp. 419–438.

\bibitem{f2}
R. Cumby, S. Figlewski, and J. Hasbrouck, \textit{Forecasting volatility and correlations with EGARCH models}, J. Derivatives 1(2) (1993), pp. 51–63.

\bibitem{f3}
S. Figlewski, \textit{Forecasting volatility}, Financ. Mark. Inst. Instrum. 6 (1997), pp. 1–88.

\bibitem{f4}
P. Jorion, \textit{Predicting volatility in the foreign exchange market}, J. Financ. 50 (1995), pp. 507–528.

\bibitem{f5}
[21] P. Jorion, \textit{Risk and turnover in the foreign exchange market, in The Microstructure of Foreign Exchange Markets}, J.A. Franke, G. Galli, and A. Giovannini, eds., Chicago University Press, Chicago, 1996.

\bibitem{f6}
D.G. McMillan, A.E.H. Speight, and O. Gwilym, \textit{Forecasting UK stock market volatility: A comparative analysis of alternate methods}, Appl. Financ. Econ. 10 (2000), pp. 435–448.

\bibitem{Vapnik1}
V.N. Vapnik, \textit{The Nature of Statistical Learning Theory}, 2nd ed., Sringer-Verlag, NewYork, 1995.


\bibitem{Vapnik2}
V.N. Vapnik, \textit{Statistical Learning Theory},Wiley, NewYork, 2001.

\end{thebibliography}