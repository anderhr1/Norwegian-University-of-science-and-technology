A list of references (a list of all papers, books or web pages referred to in the study, irrespective of whether you read them, or found them cited in other studies), as opposed to bibliography (a list of items that you read, irrespective of whether you referred to them in the study) is usually required.

\chapter{References}

\begin{thebibliography}{9}
\bibitem{B&J} 
G.E.P. Box and G.M. Jenkins, \textit{Time Series Analysis Forecasting and Control}, Holden-Day, San Francisco, 1976.

\bibitem{Engel}
R.F. Engle, \textit{Autoregressive conditional heteroscedasticity with estimates of the variance of United Kingdom inflation}, Econometrica 50 (1982), pp. 987–1007.

\bibitem{Bollerslev}
T. Bollerslev, \textit{Generalized autoregressive conditional heteroscedasticity}, J. Econ. 31 (1986), pp. 307–327.

\bibitem{Wetal}
W.C. Wong, F. Yip, and L. Xu, \textit{Financial Prediction by Finite Mixture GARCH Model}, Proceeding of Fifth International Conference on Neural Information Processing, 1998, pp. 1351–1354.

\bibitem{Tetal}
H. Tang, K.C. Chun, and L. Xu, \textit{Finite Mixture of ARMA-GARCH Model for Stock Price Prediction}, Proceedings of the 3rd International Workshop on Computational Intelligence in Economics and Finance (CIEF2003), North
Carolina, USA, 2003, pp. 1112–1119.

\bibitem{Mandelbrot}
Mandelbrot, B., \textit{The variation of certain speculative prices}, Journal of Business 36, 1963, pp. 394–419.

\bibitem{Black}
Black, F. and Litterman, R.,\textit{Global portfolio optimization}, Financial Analysts Journal, 48(5),1992, pp. 28–43.

\bibitem{A&B}
T.G.Andersen and T. Bollerslev,\textit{Answering the skeptics: Yes, standard volatility models do provide accurate forecasts}, Int. Econ. Rev. 39 (1998), pp. 885–905.

\bibitem{f1}
T.J. Brailsford and R.W. Faff, \textit{An evaluation of volatility forecasting techniques}, J. Bank. Financ. 20 (1996), pp. 419–438.

\bibitem{f2}
R. Cumby, S. Figlewski, and J. Hasbrouck, \textit{Forecasting volatility and correlations with EGARCH models}, J. Derivatives 1(2) (1993), pp. 51–63.

\bibitem{f3}
S. Figlewski, \textit{Forecasting volatility}, Financ. Mark. Inst. Instrum. 6 (1997), pp. 1–88.

\bibitem{f4}
P. Jorion, \textit{Predicting volatility in the foreign exchange market}, J. Financ. 50 (1995), pp. 507–528.

\bibitem{f5}
[21] P. Jorion, \textit{Risk and turnover in the foreign exchange market, in The Microstructure of Foreign Exchange Markets}, J.A. Franke, G. Galli, and A. Giovannini, eds., Chicago University Press, Chicago, 1996.

\bibitem{f6}
D.G. McMillan, A.E.H. Speight, and O. Gwilym, \textit{Forecasting UK stock market volatility: A comparative analysis of alternate methods}, Appl. Financ. Econ. 10 (2000), pp. 435–448.

\bibitem{Vapnik1}
V.N. Vapnik, \textit{The Nature of Statistical Learning Theory}, 2nd ed., Sringer-Verlag, NewYork, 1995.


\bibitem{Vapnik2}
V.N. Vapnik, \textit{Statistical Learning Theory},Wiley, NewYork, 2001.

\end{thebibliography}