\chapter{Introduction}

This paper examines the relationship between stocks proportion of idiosyncratic volatility of total volatility and return predictability. We consider stocks listed at the OSEAX at the Oslo stock exchange, continuously listed from 01.01.2010 to 25.09.2017. 

This  study  aims  to highlight the significance of idiosyncratic volatility on forecasting models predictability. The working hypothesis will be that the return of stocks with a high proportion of IV will be more difficult to forecast. 
 
Idiosyncratic volatility (IV) is the volatility of the stock's return explained by the firm specific risk. The contrary to idiosyncratic volatility is systematic volatility, the overall risk that affects all assets like fluctuations in the stock market or interest rates. As a result, the total volatility of a stock is the sum of the idiosyncratic risk and the systematic risk. 

Although  the  relationship between returns and idiosyncratic volatility has been  considerably addressed  in the  literature, the question of whether stocks with a high or low proportion of idiosyncratic volatility is easier to forecast has gathered less attention. Ang, Hodrick, Xing, and Zhang [2006. The cross-section of volatility and expected returns. Journal of Finance 61, 259–299], find that monthly stock returns are negatively related to the one-month lagged idiosyncratic volatilities.

Overall, in this paper we make 2 contributions to the literature  on  volatility  measures, volatility  risk and  stock  return forecasting.  First,  to our knowledge, it  is  the first research paper forecasting stocks listed on the OSEAX, sorted by idiosyncratic volatility. Second...

The results can be summarized as follows. 

The rest of the paper is organized as follows. 

- Overall, in this paper we make 2 contributions to the literature.

- This paper contains three main contributions.

- We present 3 pieces of empirical evidence consistent with our model’s predictions.

· The remainder of this paper is structured as follows