\chapter{Introduction}
The behaviour of stock returns, and the risk factors influencing them, have been extensively covered in literature. A widely accepted relationship in financial theory is the positive relationship between risk and return. This implies that investors should expect higher returns for increasing their risk.

One of the first theories describing this relationship is the widely known CAPM, expressing that all risk is covariance with the market. Other models have since been introduced, and the Fama French three factor model \cite{famafrench} extends the CAPM by also including covariance with the risk factors size and value. Common for the risk factors is the implication that idiosyncratic risk should be uncorrelated with returns, given the possibility of diversification. In contrast with these two models, Merton \cite{merton87} presented a theory implying that investors demand a higher expected return in compensation of higher idiosyncratic risk, as investors are not totally diversified. This laid the ground for the study of the relationship between idiosyncratic risk and returns.

Recently, the relationship between idiosyncratic volatility (IVOL) and future returns has been devoted much attention. Several studies, like Malkiel and Xu \cite{malkielxu02}, indicate a positive relationship between IVOL and the cross-section of expected returns. In contrast, Ang et al. \cite{angetal06}, presented results concluding that stocks with higher IVOL earns lower expected returns.

Likewise, forecasting of financial market returns has long been of interest by financial researchers and practitioners. The theory of financial market efficiency states that financial returns fully adjust to all relevant information and thus returns cannot be forecasted. However, there exists evidence of market anomalies leading to predictable patterns. For instance, Lo \cite{Lo} proposed an hypothesis according to which important changes in information or economic circumstances lead to serial return correlation. Moreover, in the Norwegian stock market, the low idiosyncratic anomaly is well covered, and Tjaum and Wiedswang \cite{thaumwiedswang} concludes that the anomaly exists, but is decreasing with time and arbitrage activity. Hence, there exists evidence of the predictability of returns.

This leads us to our study. While the return-predictive power of idiosyncratic volatility is extensively studied, there has been devoted less to none attention to the relationship between IVOL and return forecasting performance. 

In our analysis, we investigate the relationship between proportion of IVOL to volatility, defined as the IVOL percentage, and the out-of-sample, one-day-ahead return forecasting performance using a dynamical, non-linear ARMA-GARCH model. We employ a rolling window model to estimate IVOL and the out-of-sample, one-day-ahead return. In this study, we use daily financial data of selected constituents of the Oslo Stock Exchange All-Share Index (OSEAX) between January 2010 and September 2017. This study aims to contribute to the existing literature in three ways. 

Firstly, we study the relationship between an individual stock or equally sized, equally weighted portfolios sorted by the selected stocks' IVOL percentage and the out-of-sample, one-day-ahead return forecasting performance. Our purpose is to examine whether IVOL is just noise or if it contains any firm-specific information leading to patterns, making the stock or portfolio easier or more difficult to forecast. The return forecasting model performance will be evaluated in terms of both statistical and economical metrics.

Secondly, we differ from existing literature by exploring the short-term relationship between IVOL and the realized return. We are concerned with the the one-day-head realized return, while other literature typically examines the relationship in terms of realized returns on a monthly, quarterly or yearly basis.

Finally, when other literature is interested in the magnitude of the IVOL, we look at the IVOL percentage. This is because we want to infer knowledge about the return forecasting model performance based on the proportion of IVOL, not the magnitude of IVOL.

We will show that the general return forecasting performance of the ARMA-GARCH is good, across all levels of IVOL percentage. The total accumulated return generated by trading according to the daily predictions of the return forecasting model significantly outperforms the return obtained from a buy-and-hold strategy. Of equal importance, through our analysis we will show that there exists a negative relationship between IVOL percentage and return forecasting performance in terms of accuracy, precision and direction. Lastly, and perhaps most interesting, we present evidence that there is a positive relationship between IVOL percentage and returns generated by trading according to the return forecasting model, despite the negative relationship between IVOL percentage and forecasting performance in terms of accuracy, precision and direction. 

The remainder of the paper is organized as follows. In the next chapter, Chapter \ref{LR}, we will review the literature covering IVOL and financial forecasting models. Chapter \ref{Data} explains the data used and the data cleaning process. In Chapter \ref{Methodology}, we will try to connect the two literature spaces reviewed in Chapter \ref{LR}, with our forecasting model and evaluation framework. Chapter \ref{Results} presents the results of our analysis and Chapter \ref{Conclusions} the conclusion. Finally, in Chapter \ref{FW}, we have devoted a whole chapter to future works as we have many ideas on how to improve our return forecasting performance. In addition, there are still unanswered questions concerning the relationship between IVOL percentage, and the out-of-sample, one-day-ahead return forecasting performance of ARMA-GARCH.

