\chapter{Introduction}
The behaviour of stock returns, and the risk factors influencing them, have been extensively covered in literature. One of the most widely accepted relations
within the field of finance is the positive relationship between risk and return. Bearing risk can be expected to produce a reward in form of higher expected returns.

The CAPM expresses that all risk is covariance with the market, while the Fama French three factor model \cite{famafrench} extends the CAPM by also including covariance with the risk factors size and value. In contrast with these models, Merton \cite{merton87} presented a theory implying that investors demand a higher expected return in compensation of higher idiosyncratic risk, as investors are not totally diversified. This layed the ground for the study of the relationship between idiosyncratic risk and returns.

Recently, the relationship between idiosyncratic volatility and future returns has been devoted much attention. Several studies, like Malkiel and Xu \cite{malkielxu02}, indicate a positive relationship between IV and the cross-section of expected returns. In contrast, Ang et al. \cite{angetal06}, presented groundbreaking results concluding that stocks with higher IV earned lower expected returns.

Although the return-predictive power of idiosyncratic volatility is extensively studied, there has been devoted less attention to the relationship between IV and return forecasting performance. 

In our analysis, we investigate the relationship between proportion of IV to volatility, called the IV percentage, and the out-of-sample one-day-ahead return forecasting performance using a dynamical ARMA-GARCH. We employ a rolling window
model to estimate idiosyncratic volatility and the one-day-ahead return. In this study, we use daily financial data of selected constituents of the Oslo Stock Exchange All-Share Index (OSEAX) between January 2010 and September 2017. This paper aims to contribute to the existing literature in three ways. 

Firstly, we study the relationship between an individual stock or an equally sized, equally weighted portfolio pooled by their constituents IV percentage and the one-day-ahead return forecasting performance. Our purpose is to examine whether idiosyncratic volatility is just noise or if it contains any firm-specific information leading to patterns, making the stock or portfolio easier or more difficult to forecast. The return forecasting performance will be evaluated in terms of both statistical and economical metrics.

Secondly, we differ from existing literature by exploring the short-term relationship between IV and the realized return. We are concerned with the the one-day-head realized return, while other literature typically examines the relationship in terms of realized returns on a monthly, quarterly or yearly basis.

Finally, when other literature is interested in the magnitude of the idiosyncratic volatility, we look at the idiosyncratic volatility percentage, the proportion of IV to volatility. This is because we want to infer knowledge about the return forecasting performance based on the proportion of IV and not the magnitude.

Through our analysis we will show that it exists a negative relationship between IV percentage and return forecast performance in terms of accuracy, precision and direction. Of equal importance, we will show that the return forecasting performance of the ARMA-GARCH is good. Trading according to the return forecasting model generates returns better than those obtained by following a simple buy-and-hold strategy, across all levels of IV percentage. Finally, and perhaps most interesting, we present evidence that there is a positive relationship between IV percentage and returns generated from the return forecasting model, despite the negative relationship between IV percentage and forecast performance in terms of accuracy, precision and direction. 

The remainder of the paper is organized as follows. In the next chapter, Chapter \ref{LR}, we will review literature covering idiosyncratic volatility and financial forecasting models. Chapter \ref{Da} briefly explains our data used and the data cleaning process. In Chapter \ref{Methodology}, we will try to connect the two literature spaces reviewed in Chapter \ref{LR} with our forecasting model and testing framework. Chapter \ref{Results} presents the results of our analysis and Chapter \ref{Conclusions} concludes. Finally, in Chapter \ref{FW}, we have devoted a whole chapter to future works as we have many ideas on how to improve our return forecasting performance. In addition, there are still unanswered questions concerning the relationship between IV percentage, and the out-of-sample one-day-ahead return forecasting performance of ARMA-GARCH.
