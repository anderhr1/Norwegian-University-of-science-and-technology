\chapter{Abstract}
This study examines the relationship between daily idiosyncratic volatility (IVOL) percentage and the out-of-sample one-day-ahead return forecasting performance using a dynamical ARMA-GARCH. We examine both individual stocks and equally sized, equally weighted portfolios sorted by the selected stocks' IVOL percentage. The IVOL and the one-day-ahead return forecast is estimated using a rolling window model. The return forecasting model performance is evaluated in terms of both statistical and economical metrics.

Firstly, we show that the general return forecasting performance of the ARMA-GARCH is good, across all levels of IVOL percentage. Considering the total accumulated return generated by trading according to the daily predictions of return forecasting model over the period considered, it significantly outperforms the simple buy-and-hold strategy.

Secondly, our results indicate a negative relationship between daily IVOL percentage and forecasting performance in terms of precision, accuracy and direction. The results indicate that it is more difficult to predict the one-day-ahead return for stocks with a high average daily IVOL percentage.

Thirdly, and perhaps most interesting, we present evidence that there is a positive relationship between IVOL percentage and accumulated returns generated from trading according to the daily predictions of the return forecasting model, despite the negative relationship between daily IVOL percentage and forecasting performance in terms of accuracy, precision and direction.

