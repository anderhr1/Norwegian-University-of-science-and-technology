\chapter{Abstract}
This study examines the relationship between daily idiosyncratic volatility percentage and out-of-sample one-day-ahead return forecasting performance using a dynamical ARMA-GARCH, for both individual stocks and equally sized, equally weighted portfolios sorted by their constituent's IV percentage. We employ daily rebalancing, using a rolling window model to estimate idiosyncratic volatility and the one-day-ahead return. The return forecasting performance is evaluated in terms of both statistical and economical metrics.

Firstly, we show that the return forecasting performance of the ARMA-GARCH is good. Trading according to the return forecasting model generates returns better than those obtained by following a simple buy-and-hold strategy, across all levels of IV percentage. 

Secondly, our results indicate a negative relationship between daily IV percentage and return forecasting performance in terms of accuracy, precision and direction. 

Thirdly, and perhaps most interesting,  we present evidence that there is a positive relationship between IV percentage and returns generated from the return forecasting model, despite the negative relationship between IV-percentage and forecast performance in terms of accuracy, precision and direction.

%Finally, our findings contribute to the existing literature by reconciling the mixed results for the relationship between idiosyncratic volatility and mispricing displayed in the previous literature.
%MÅ SKRIVES OM TIL VÅR OPPAVE
