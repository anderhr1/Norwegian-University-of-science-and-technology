\begin{abstract}
    
This study examines the relationship between the proportion of idiosyncratic volatility (IVOL) to volatility, defined as the IVOL percentage, and the out-of-sample, one-day-ahead return forecasting performance using a dynamical ARMA-GARCH model for selected stocks at the Oslo Stock Exchange All-share Index (OSEAX). We examine both individual stocks and equally sized, equally weighted portfolios sorted by the selected stocks' IVOL percentage. The model employs a rolling window approach to estimate the IVOL and the one-day-ahead return forecast. The return forecasting performance is evaluated in terms of both statistical and economical metrics.

Firstly, we show that the general return forecasting performance of the ARMA-GARCH is good, across all levels of IVOL percentage. The total accumulated return generated by trading according to the daily predictions of the return forecasting model significantly outperforms the return obtained from a buy-and-hold strategy.

Secondly, our results indicate a negative relationship between daily IVOL percentage and forecasting performance in terms of precision, accuracy and direction. Hence, it is more difficult to predict the one-day-ahead return for stocks with a high average daily IVOL percentage.

Thirdly, and perhaps most interesting, we present evidence that there is a positive relationship between daily IVOL percentage and accumulated returns generated from trading according to the daily predictions of the return forecasting model, despite the negative relationship between daily IVOL percentage and forecasting performance in terms of accuracy, precision and direction.

\end{abstract}