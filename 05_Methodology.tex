\chapter{Methodology}

The methodology of this paper is structured with four main components. In the first part of the study we calculate the daily idiosyncratic volatility (IV) for each of the selected stocks by applying the \cite{famafrench} regression, updating the estimate each day based on the last 250 days in a rolling window fashion. We do this for each day over the whole period.

Secondly, we segment the stocks into four equally sized portfolios by their proportion of daily IV to economic standard deviation (the standard deviation of the realized returns), which we denote as the daily IV percentage. We implement daily rebalancing of the portfolios.

Thirdly, we build an ARMA-GARCH forecasting model. We estimate the ARMA-GARCH parameters for each stock each day based on the 250 previous daily returns, and construct the one-day-ahead out-of-sample forecasts, over the whole period used as rolling window. We then roll the window one day forward. We compare the model’s one-day-ahead forecast with the realized return to obtain forecasting error. This is assigned to its associated portfolio and used for revising the forecasting model. 

Lastly, we revise the goodness-of the forecasting by applying statistical and economical metrics of comparison, which will be described in further detail.

\section*{Calculating the idiosyncratic volatility and creation of portfolios segmentation by daily IV percentage}

Consistent with Ang et al. (2006) we define idiosyncratic volatility as the square root of the variance of the error term in the Fama French Three factor model.

\subsection*{Regression based on the Fama French three factor model}  This model incorporates the empirical fact that value and small-cap stocks have a tendency to outperform the market on a regular basis. Using this methodology we have the following multiple linear regression:
\begin{align} 
    r_{i,t} - r_{f,t}= \alpha_{i,t} + \beta_{m,i,t}(r_{m,t} - r_{f,t}) + \beta_{SMB,i,t}r_{SMB,t} + \beta_{HML,i,t}r_{HML,t} + \epsilon_{i,t}, \quad  \forall i \in I \quad  \forall t \in T 
    \label{FFregression}
\end{align}
where $t$ indicates the day in the sample period, $r_{i,t}$ is the daily realized return of the stock $i$, $r_{f,t}$ is defined as the daily risk-free rate of Norwegian 10-years government bonds, $r_{m,t}$ is the daily return of the OSEAX, $r_{SMB,t}$ is the daily return of small cap stocks minus large cap stocks, $r_{HML,t}$ is the daily return of stocks that have $\frac{B}{M}$ ratio above average minus the ones with the ratio below average. $\beta_{m,i,t}$, $\beta_{SMB,i,t}$ and $\beta_{HML,i,t}$ is the corresponding coefficients resulting from the regression and the regression error is $\epsilon_{i,t}$.

More precisely, the $r_{m,t}$, $r_{HML,t}$ and $r_{SMB,t}$ are defined as follows:
\begin{align}
    r_{m,t} &= \sum_{i=1}^{N} r_{i,t} \cdot \frac{\text{MC}_{i,t}}{\sum_{s=1}^{N} MC_{s,t}},  \quad  \forall t \in T \\
    r_{SMB,t} &= \sum_{i=1}^{\frac{N}{2}} r_{i,t} \cdot \frac{\text{MC}_{i,t}}{\sum_{s=1}^{\frac{N}{2}} MC_{s,t}} - \sum_{k=\frac{N}{2}+1}^{N} r_{k,t} \cdot \frac{\text{MC}_{k,t}}{\sum_{s={\frac{N}{2}+1}}^{N} MC_{s,t}}, \quad  \forall t \in T \\
    r_{HML,t} &= \sum_{i=1}^{\frac{N}{2}} r_{i,t} \cdot \frac{\text{MC}_{i,t}}{\sum_{s=1}^{\frac{N}{2}} MC_{s,t}} - \sum_{k=\frac{N}{2}+1}^{N} r_{k,t} \cdot \frac{\text{MC}_{k,t}}{\sum_{s=\frac{N}{2}+1}^{N} MC_{s,t}} \quad  \forall t \in T
\end{align}
where $N$ is the number of stocks in the sample, $r_{i,t}$ and $MC_{i,t}$ is the realized return and market capitalization of stock $i$ at day $t$, respectively. 

The daily return of the market was obtained by weighting each of the XX stocks in the sample according to their market values. The excess daily market return was calculated as the difference between the return of the market and the converted daily rate of Norwegian 10-years government bonds. \cite{famafrench} methodology was used to calculate the daily returns of the SMB and HML portfolios.

The regression where run on the return series for each stock on each day starting at day equal to the rolling window size, 250, in accordance to \ref{FFregression}. 

\subsection*{Definition of idiosyncratic volatility}
Consistent with Ang et al. (2006) we define idiosyncratic volatility as the square root of the variance of the error term in the Fama French Three factor model. Hence, from equation \ref{FFregression}, we have:
 \begin{align}
 IV_{i,t} &= \sqrt{Var[\epsilon_{i,t}]} \\
  &= \sqrt{Var[r_{i,t} - r_{f,t}- \alpha_{i,t}-\beta_{m,i,t}(r_{m,t} - r_{f,t}) - \beta_{SMB,i,t}r_{SMB,t} - \beta_{HML,i,t}r_{HML,t}]},  \quad  \forall i \in I \quad  \forall t \in T \\
 &= \sqrt{Var[r_{i,t}]+\beta_{1,i,t}^2Var[r_{m,t}]+\beta_{SMB,t}^2Var[r_{SMB,t}]+\beta_{HML,t}^2Var[r_{HML,t}]}, \quad  \forall i \in I \quad  \forall t \in T \\ 
 &= \sqrt{\sigma_{i,t}^2 - \beta^2_{m,i,t} \sigma_{m,t}^{2}- \beta^2_{SMB,t} \sigma_{SMB,t}^{2}- \beta^2_{3,i,t} - \sigma_{HML,t}^{2}, \quad  \forall i \in I \quad  \forall t \in T
 \end{align}
where $t$ is the current period, T is the number of days the rolling window runs over and $IV_{i,t}$ is the IV of stock $i$ in period $t$.

Having estimated the daily IV values for all the selected stocks, we calculate the daily IV percentage. We have chosen to use the IV percentage to control for 

and segment the stocks into four equally sized portfolios, by increasing proportion of daily IV percentage. We implement daily rebalancing of the buckets, \textbf{and weight the stocks equally. }

\section*{Building the forecasting model: ARMA($w,x$)-GARCH($y,z$)} 

Financial returns are often modelled as auto-regressive moving average (ARMA) time series with random disturbances having conditional heteroscedastic variances. The conditional mean and conditional variance will change at every point in time because it depends on the history of returns up to that point. That is, we account for the dynamic properties of returns by regarding their distribution at any point in time as being conditional on all the information up to that point. The distribution of a return at time t regards all the past returns up to and including time $t-1$ as being non-stochastic. We denote the information set, which is the set containing all the past returns up to and including time $t-1$, by $I_{t-1}$. The information set contains all the prices and returns that we can observe, like the filtration set in continuous time. 

We write $\sigma_t^2$ to denote the conditional variance at time $t$. This is the variance at time $t$, conditional on the information set. That is, we assume that everything in the information set is not random because we have an observation on it. When the conditional distributions of returns at every point in time are all normal we write:
\begin{align}
    r_t | I_{t-1} \sim N(0,{\sigma_t^2})
\end{align}

Often one would choose other distributions due to the fact that financial
series often have fat tails. One option would be to use the Student-t
distribution distribution or a skewed version of it. This paper assumes that returns are having a normal distribution. 

\subsection*{The Conditional Mean Equation, ARMA($w$,$x$)}

For modeling data series we used two common concepts of conditional mean: the auto regressive (AR) process and the moving average (MA) process. Together the two processes constitute the conditional mean equation. 

The AR process is given by:
\begin{align}
    r_{i,t}=c_i + \sum_{j=1}^w\kappa_j r_{i,t-j} + \epsilon_{i,t},\quad \epsilon_{i,t} | I_{i,t-1} \sim N(0,{\sigma_{i,t}^2}) \label{ConditionalMeanEquation}
\end{align}
where $\kappa_j$ is the lag parameter of the observed variable, $r_t$ is the random observed variable at time $t$ dending on the previously realized values of $r_{t-j}$, $c$ is the mean constant and $\epsilon_t$ the white noise.

The MA process is given by:
\begin{align}
    r_{i,t}=c_i + \sum_{j=1}^x\mu_{i,j} \epsilon_{i,t-j} + \epsilon_{i,t},\quad \epsilon_{i,t} | I_{i,t-1} \sim N(0,{\sigma_{i,t}^2}) \label{ConditionalMeanEquation}
\end{align}
where $\mu_j$ is the lag parameter of the observed variable, $r_t$ is the random observed variable at time $t$ depending on the previously realized values of $\epsilon_{t-j}$, $c$ is the mean constant and $\epsilon_t$ the white noise.

The combination of both the AR-process and MA-process, gives us the ARMA process described by:
\begin{align}
    r_{i,t}=c_i+\sum_{j=1}^w\kappa_{i,j} r_{i,t-j}+\sum_{j=1}^x\mu_{i,j} \epsilon_{i,t-j}+\epsilon_{i,t},\quad \epsilon_{i,t} | I_{i,t-1} \sim N(0,{\sigma_{i,t}^2}) \label{ConditionalMeanEquation}
\end{align}

\subsection*{The Conditional Variance Equation, GARCH($y,z$)}
As financial data time series usually exhibit volatility clustering, a model dealing with
conditional heteroskedasticity should be considered. We use the GARCH model introduced by (Bollerslev, 1986),which is a generalization of the ARCH model that was originally developed by (Engle, 1982). The ARCH model allows for long lags in conditional variance and the GARCH model extends it in the way that it allows for both long lags in conditional variance and a more flexible lag structure.

The GARCH($y$,$z$) has the conditional volatility equation given by:

\begin{align}
    \sigma_{i,t^2} &= \omega_i + \sum_{j=1}^y\alpha_{i,j}\epsilon_{i,t-j}^2+\sum_{j=1}^z\beta_{i,j}\sigma_{i,t-j}^2,\quad\epsilon_{i,t} | I_{i,t-1} \sim N(0,{\sigma_{i,t}^2}) \label{ConditionalVolatilityEquation}
\end{align}

The GARCH error parameter $\alpha$ measures the reaction of conditional volatility to market shocks. When $\alpha$ is relatively large, above 0.1, the volatility is very sensitive to market events.

The GARCH lag parameter $\beta$ measures the persistence in conditional volatility irrespective of anything happening in the market. When $\beta$ is relatively large, above 0.9, then volatility takes a long time to die out.

\subsection*{Long term volatility}

In the absence of market shocks the GARCH variance will eventually settle down to a steady state value. This is the value $\bar{\sigma}^2$ such that ${\sigma_t^2} = \bar{\sigma}^2$ for all t. We call $\bar{\sigma}^2$ the unconditional variance of the GARCH model. It corresponds to a long term average value of the conditional variance. The theoretical value of the GARCH long term or unconditional variance is not the same as the unconditional variance in a moving average volatility model. The moving average unconditional variance is called the i.i.d. variance because it is based on the i.i.d. returns assumption. The theoretical value of the unconditional variance in a GARCH model is clearly not based on the i.i.d. returns assumption. In fact, the GARCH unconditional variance differs depending on the GARCH model. The long term or unconditional variance is found by substituting ${\sigma_t^2} = {\sigma_{t-1}^2} = \bar{\sigma}^2$ into the GARCH conditional variance equation.We also use the fact that $E(\epsilon_{t-1}^2)=\sigma_{t-1}^2$. This yields the following formula for the long term variance of the GARCH model:

\begin{align}
    \bar{\sigma}_i^2=\frac{\omega_i}{1-(\sum_{j=1}^y\alpha_{i,j}+\sum_{j=1}^z\beta_{i,j})} \label{longTermVolatilityGARCH}
\end{align}

\subsection*{Parameter Estimation}

The plain vanilla ARMA and ARMA-GARCH parameters are estimated by maximizing the value of the log likelihood function. As mentioned earlier, in this paper we assume that the distribution of the error process is normal with expectation 0 and conditional variance ${\sigma_t^2}$. Furthermore, we have assumed stationarity, so the unconditional variance, in the case of plain vanilla ARMA, is ${\bar\sigma^2}$. With these assumptions in mind, we can use the normal log likelihood function.
\subsubsection{Plain Vanilla ARMA($w,x$)}
Maximizing the ARMA likelihood reduces to the problem of maximizing:
\begin{align} 
    ln(L_{i,t})=-\frac{1}{2}\sum_{t=1}^T\bigg( ln(\sigma_{i}^2)+(\frac{\epsilon_{i,t}}{\sigma_i})^2\bigg)  \label{MaximumLikeARMA}
\end{align}
with respect to all the parameters. To do this we solve the conditional mean equation (\ref{ConditionalMeanEquationARMA}) for $\epsilon_t$:
\begin{align}
     \epsilon_{i,t}=r_{i,t}-\sum_{j=1}^w\kappa_{i,j} r_{i,t-j}-\sum_{j=1}^x\mu_{i,j} \epsilon_{i,t-j}-c_i \label{ConditionalMeanEquationOnEpsilon}
\end{align}
Finally we insert the above equation (\ref{ConditionalMeanEquationOnEpsilonARMA}) and the conditional volatility equation (\ref{ConditionalVolatilityEquation}) into the maximum likelihood function (\ref{MaximumLikeARMA}):
\begin{align} 
    ln(L_{i,t})=-\frac{1}{2}\sum_{t=1}^T\Bigg( ln(\sigma_i^2)+\Big(\frac{(r_{i,t}-\sum_{j=1}^w\kappa_{i,j} r_{i,t-j}-\sum_{j=1}^x\mu_{i,j} \epsilon_{i,t-j}-c_i)^2}{\sigma_i^2}\Big)\Bigg)  \label{fullMaximumLikeARMA}
\end{align}

\subsubsection{ARMA($w,x$)-GARCH($y,z$)}
Maximizing the ARMA-GARCH likelihood reduces to the problem of maximizing:
\begin{align} 
    ln(L_{i,t})=-\frac{1}{2}\sum_{t=1}^T\bigg( ln(\sigma_{i,t}^2)+(\frac{\epsilon_{i,t}}{\sigma_{i,t}})^2\bigg)   \label{MaximumLike}
\end{align}
with respect to all the parameters. The conditional mean equation is solved on epsilon, just as in equation \ref{ConditionalMeanEquationOnEpsilon}. The equation is then inserted, together with the conditional volatility equation (\ref{ConditionalVolatilityEquation}), into the maximum likelihood function (\ref{MaximumLike}):
\begin{align} 
    ln(L_{i,t})=-\frac{1}{2}\sum_{t=1}^T \Bigg(ln\Big(\omega_i + \sum_{j=1}^y\alpha_{i,j}\epsilon_{i,t-j}^2+\sum_{j=1}^z\beta_{i,j}\sigma_{i,t-j}^2\big)+\Big(\frac{(r_{i,t}-\sum_{j=1}^w\kappa_{i,j} r_{i,t-j}-\sum_{j=1}^x\mu_{i,j} \epsilon_{i,t-j}-c_i)^2}{\omega_i + \sum_{j=1}^y \alpha_{i,j} \epsilon_{i,t-j}^2 +\sum_{j=1}^z\beta_{i,j}\sigma_{i,t-j}^2}\Big)\Bigg)   \label{fullMaximumLike}
\end{align}
The parameter constraints are:
\begin{align} 
    \omega_i>0,\quad\quad \alpha_{i,j},\beta_{i,j}\geq0 \quad \forall j, \quad \sum_{j=1}^y\alpha_{i,j}+\sum_{j=1}^z\beta_{i,j}<1 \label{ParameterConstraints}
\end{align}

\subsection*{Implementation of ARMA($w,x$)-GARCH($1,1$)}
To forecast returns we will try to fit an auto regressive moving average return model (ARMA) of order ($w,x$) with asymmetric generalized auto-regressive conditional hetereoscadasticity (GARCH) of order ($1$,$1$). The estimation of the ARMA($w,x$)-GARCH($1,1$) parameters is done in R, while the rest of the calculations in this paper is done in Python. The reason we chose to calculate the parameters of the ARMA($w,x$)-GARCH($1,1$) in R, is because there are currently no packages in Python supporting ARMA-GARCH, only AR-GARCH. 

The first thing to be decided, is the ARMA order. In other words, we have to decide which values the parameters, $w$ and $x$, in the conditional mean equation (\ref{ConditionalMeanEquation}) should have. The parameters and hence the maximum likelihood is calculated for all combinations of $w$ and $x$, using equation \ref{fullMaximumLikeARMA}, where $w\in[0,3]$ and $x\in[0,3]$. The final values of $w$ and $x$, the ARMA order, to use in our forecast is chosen based on the the Akaike information criterion (AIC). AIC is an estimator of the relative quality of statistical models for a given set of data. Given a collection of models for the data, AIC estimates the quality of each model, relative to each of the other models. Thus, AIC provides a means for model selection. The AIC of a given model is calculated using the following equation:
\begin{align}
    AIC_{i,t}=2k_i-2ln(L_{i,t})
\end{align}
where $k$ is the number of estmated parameters and $L$ is the maximum value of the likelihood function. 

After the decision is made of which ARMA order to use, equation (\ref{fullMaximumLike}) is solved. There is no guarantee that the maximum likelihood of the ARMA-GARCH will converge. To increase the probability of convergence, the algorithm in R tries four different solvers. If there is no convergence, the conditional mean parameters are found using plain vanilla ARMA (\ref{fullMaximumLikeARMA}) and not ARMA-GARCH (\ref{fullMaximumLike}). The maximum likelihood function of plain vanilla ARMA will almost always converge because it is linear. If the plain vanilla ARMA does not converge, the process above is repeated with the ARMA order giving second best AIC, and so on, until convergence is achieved.

The ARMA($w,x$)-GARCH($1,1$) parameters are estimated based on the information set for each stock, $i$, for every time step, over the whole period used as a rolling window. The time step is one day and the information set, sample, which the parameters are estimated from are the past 250 daily returns. As mentioned in the introduction to this chapter, there are 107 stocks, the period used as a rolling window is from January 2011 to September 2017, constituting about 1800 trading days. The model setup results in a huge number of needed calculations, and as a consequence, we were granted access to the calculation cluster Solstorm belonging to the Department of Industrial Economics and Technology Management at NTNU. 

One may argue that a information set bigger than 250 should be used, as the standard errors of the parameters are only valid asymptotically. This, additional forecasting model spesification and other forecasting models are further discussed in Chapter \ref{FutureWork}.


\subsection*{Return Forecasting}

After the model in the previous subsection is run on the data, we obtain optimal values for all the parameters for each stock every day. To forecast the the one-day-ahead return, and hence test our out-of sample performance, we need to compute the conditional expected return of the conditional mean equation (\ref{ConditionalMeanEquation}). The expected return at time $t$ for a given stock, $i$, is:
\begin{align} 
    E(r_{i,t})=c_i+\sum_{j=1}^w\kappa_{i,j} r_{i,t-j}+\sum_{j=1}^w\mu_{i,j} \epsilon_{i,t-j} \label{ExpectedConditionalMean}
\end{align}
The expectation of the error process, $\epsilon_t$, is assumed to be zero. The parameters obtained in the previous subsection is our best guess on what the parameters would be tomorrow. Hence, the optimal parameters at time $t$ is used to forecast the return at time $t+1$. Using equation \label{ExpectedConditionalMean}, our forecast at time $t$ for time $t+1$ for a given stock,$i$, is:
\begin{align} 
    E(r_{i,t+1})=c_i+\sum_{j=0}^w\kappa_{i,j} r_{i,t-j}+\sum_{j=0}^x\mu_{i,j} \epsilon_{i,t-j}
\end{align}

\section*{Revise the goodness-of the forecasting}
Having run our forecasting model and produced a series of forecasting estimates, we calculate the forecasting error for each day in the out-of-sample period. We define the forecasting error as:
\begin{align}
    \epsilon_{i,t}^{f} = r_{i,t}^{r} - r_{i,t}^{e}
\end{align}
where $\epsilon_{i,t}^{f}$ is the forecasting error, the difference between the forecasting estimate, $r_{i,t}^{e}$, and realized returns, $r_{i,t}^{r}$, for all stocks and all time steps.

In order to assess the return forecasting accuracy, we apply two separate statistical metrics and an economical metrics to measure the profitability of the following the forecasting model 

We define the root mean square error metric and a sign metric as our statistical metrics. Firstly, we have the root mean square error metric (RMSE), that is the standard deviation of the residuals, which is defined as:
\begin{align}
    RMSE_{\tau} = \sqrt{\frac{\sum_{i=1}^{N}(r_{i,t}^{r} - r_{i,t}^{e})^{2}}{n}}, \quad \forall \tau in \Sigma
\end{align}
where $i$ is stock and $\tau$ is the current period and $\Sigma$ is the total number of periods. We calculate the RMSE for each portfolio and each period, summing the total RMSE for each portfolio over the total sample period, before finding the average by dividing by number for periods:
\begin{align}
    RMSE_{b} = \frac{1}{|\Sigma|}\sum_{\tau=1}^{|\Sigma|}\sqrt{\frac{\sum_{i=1}^{N}(r_{i,t}^{r} - r_{i,t}^{e})^{2}}{n}}, \quad \forall b \in B
\end{align}

Further, as a measure of the market timing ability of the forecasting model, we calculate the proportion of times the model correctly predicted the direction of the return. We define this percentage as Sign and its corresponding return as $r_{sign}$. 

Moreover, we also apply economical metrics. We wish to examine the profitability of following the predictions of our forecasting model, compared to an ordinary buy-and-hold strategy. 

Hence, we define sign strategy as  following the one-day-ahead predictions from the forecasting model from the start to the end of the forecasting period. That means taking a long position if the model predicts a positive return, and a short position if the model predicts a negative return for the next time step. The annualized return of this strategy is defined as the sign return, $r_{sign}$.
If the prediction is correct, the realized return is assigned to the $r_{sign}$. Likewise, if the prediction is wrong, the realized return is withdrawn from the accumulated return, $r_{sign}$.

Moreover, we define the economic return, $r_{economic}$, as the annualized return obtained from following a buy-and-hold strategy, holding a stock from the start until the end of the period.

Further, we define alpha, $\alpha$, as the difference $r_{sign}-r_{economic}$. This is the excess return from following the sign strategy compared to the buy-and-hold strategy. 
