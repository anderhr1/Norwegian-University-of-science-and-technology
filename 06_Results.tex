The result section will usually be tabulated or graphed, and each table or figure should be described, noting any interesting features – whether expected or unexpected, and in particular, inferences should relate to the original aims and objectives of the research outlined in the introduction. Results should be discussed and analysed, not simply presented blandly. Comparisons should also be drawn with the results or similar existing studies if relevant – do your results confirm or contradict those of previous research? Each table or figure should be mentioned explicitly in the text. Do not include in the project any tables or figures which are not discussed in the text. It is also worth trying to present the results in as interesting and varied way as possible – for example, including figures and charts as well as just tables.


\chapter{Results}