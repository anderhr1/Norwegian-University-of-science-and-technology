\chapter{Future Work} \label{FutureWork}
\label{FW}
We have devoted a whole chapter to future works as we have many ideas on how to leverage our initial results and to improve our return forecasting performance. In addition, there are still unanswered questions concerning the relationship between IV percentage and the out-of-sample one-day-ahead return forecasting performance of ARMA-GARCH.

The cause of the positive relationship between daily IV percentage and return generated from the forecasting model, despite the negative relationship between daily IV percentage and forecast performance in terms of accuracy, precision and direction, is left somewhat unanswered. As we have already pointed out, remembering that stocks and portfolios with higher daily IV percentage also tend to have higher economic standard deviation, it might be easier for the forecasting model to predict the bigger, directional movements. Other explanations of the observed relationship might be the choice of using equally weighted portfolios, which is daily rebalanced (using daily financial data). It is difficult to make any conclusions without comparing different return forecasting models, such as artificial intelligence-based forecasting models, and types of portfolios. In addition, we need to do comparisons with different financial data granularity. 

In addition to the aforementioned comparisons, we wish to test if we can improve the return forecasting performance by using other sample sizes. In the case of ARMA-GARCH, one may argue that a sample size bigger than 250 should be used, as the standard errors of the parameters are only valid asymptotically. However, there is a trade-off between long term relationships and momentum. Another interesting feature would be to include more lags in the conditional variance equation. We would also like to examine our results with different assumptions about the underlying daily return distribution. Often one would choose other distributions due to the fact that financial series often have fat tails, as is the case with our data, showed in Chapter \ref{Da}. One option would be to use the Student-t distribution distribution or a skewed version of it.

Including transaction costs and adding trading volume constraints are also features we wish to add in the future. In a worst case scenario, the forecasting model trades on every stock every day, which implies significant operating expenses. The trading volume constraint will ensure that the stocks are possible to long and short each day. 

it would also be interesting to create portfolios that are based on quantiles of IV percentage, instead of equal size. 

Finally, with several thousand lines of Python and R code, one can never exclude the possibility of coding errors. However, we feel confident that our testing and debugging processes have been satisfactory. 