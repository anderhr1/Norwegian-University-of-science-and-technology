\chapter{Future Work} \label{FutureWork}
\label{FW}
We have devoted a whole chapter to future work as we have many ideas on how to leverage our initial results and to improve our return forecasting performance. In addition, there are still unanswered questions concerning the relationship between IVOL percentage and the out-of-sample, one-day-ahead return forecasting performance of the ARMA-GARCH model.

The cause of the positive relationship between daily IVOL percentage and returns generated from trading according to the predictions of the return forecasting model, despite the negative relationship between daily IVOL percentage and forecast performance in terms of accuracy, precision and direction, is left somewhat unanswered. As we have already pointed out, it is difficult to make any conclusions without comparing different return forecasting models, such as artificial intelligence-based forecasting models, rebalancing interval and types of portfolios. In addition, we need to do comparisons with different financial data granularity, such as weekly or monthly data.

In addition to the aforementioned comparisons, we wish to test if we can improve the return forecasting performance by using other sample sizes. In the case of ARMA-GARCH, one may argue that a sample size bigger than 250 should be used, as the standard errors of the parameters are only valid asymptotically. However, there is a trade-off between long term relationships and momentum. Another interesting feature would be to include more lags in the conditional variance equation. We would also like to examine our results with different assumptions about the underlying daily return distribution than the normal distribution we employ. Often one would choose other distributions due to the fact that financial series often have fat tails, as is the case with our data, showed in Chapter \ref{Data}. One option would be to use the Student-t distribution distribution or a skewed version of it.

Lastly, but maybe most important, we wish to include transaction costs and trading volume constraints. In a worst case scenario, the forecasting model trades on every stock, every day. This implies significant operating expenses. The trading volume constraint will ensure that the stocks are possible to long and short each day. 

